\documentclass[12pt,letterpaper]{article}
\usepackage{graphicx}

\usepackage[margin=1in]{geometry}

\begin{document}

\begin{flushright}
Homework 3: Profile Guided Optimization\\
Paul Vines and Eric Mullen\\
CSE 501\\
\end{flushright}

\subsection*{Optimizations}
\subsubsection*{Dynamic Type Refinement}
WRITE ME PAUL WRITE ME WRITE ME WRITE ME DON'T FORGET TO WRITE ME BEFORE YOU TURN ME IN
\subsubsection*{Code Layout}
Here, we aim to improve the layout of basic blocks in memory by
counting how many times branches are taken in the code. First, we
instrument all unconditional branches with a counter, and instrument
each conditional branch with two counters. We do this while the code
is in SSA form, which is also the point in our code where all
transitions between basic blocks are explicit, i.e. there are no
points where we simply fall off the end of a basic block into the next
basic block. These counters are then used as the priorities of the
basic blocks when they are layed out in each function. Previously,
blocks were layed out in an arbitrary topological sort, now they are
layed out in a priority topological sort. The only difference is,
instead of a worklist, a priority queue is used to keep the blocks
currently being processed, and thus higher priority blocks (those on
the hot code path) are all layed out close to each other.

This has surprisingly little effect on runtime, in most cases leaving
the optimized program with the exact same dynamic cycle count as its
non-optimized version.


\end{document}
