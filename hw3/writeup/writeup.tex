\documentclass[12pt,letterpaper]{article}
\usepackage{graphicx}

\usepackage[margin=1in]{geometry}

\begin{document}

\begin{flushright}
Homework 3: Profile Guided Optimization\\
Paul Vines and Eric Mullen\\
CSE 501\\
\end{flushright}

\subsection*{Optimizations}
\subsubsection*{Dynamic Type Refinement}
WRITE ME PAUL WRITE ME WRITE ME WRITE ME DON'T FORGET TO WRITE ME BEFORE YOU TURN ME IN
\subsubsection*{Code Layout}

This optimization is enabled with the option ``cbr''. It will not work
if ``ssa'' is not also specified.

Here, we aim to improve the layout of basic blocks in memory by
counting how many times branches are taken in the code. First, we
instrument all unconditional branches with a counter, and instrument
each conditional branch with two counters. We do this while the code
is in SSA form, which is also the point in our code where all
transitions between basic blocks are explicit, i.e. there are no
points where we simply fall off the end of a basic block into the next
basic block. These counters are then used as the priorities of the
basic blocks when they are layed out in each function. Previously,
blocks were layed out in an arbitrary topological sort, now they are
layed out in a priority topological sort. The only difference is,
instead of a worklist, a priority queue is used to keep the blocks
currently being processed, and thus higher priority blocks (those on
the hot code path) are all layed out close to each other.

This has surprisingly little effect on runtime, in most cases leaving
the optimized program with the exact same dynamic cycle count as its
non-optimized version. In the cases where the runtime is not the same,
the optimized code ends up running slightly slower. We are not
entirely sure what causes this. These results confirmed our hypothesis
that code layout was not an especially large factor in program
runtime, and that the original order a program is written in is in
fact not that bad as a basic block layout.


\subsubsection*{Output Memoization}

This optimization is enabled by the ``mem'' option.

This optimization runs the program, and grabs all output the program
produces. It then constructs a program that is a series of
instructions that writes the same output, and produces that as its
optimized program.

This optimization is provably optimal, if we consider program
equivalence as only an I/O equivalence, i.e. two programs are
equivalent if and only if they produce the same output. We noticed the
key point that Start has no capacity for input (other than constants
hard coded into the program text), and has only the capacity to write
integers and newline characters to the screen. Thus, if we have some
input program that produces some output when run, there is an infinite
set of programs that produce the same output when run. However, one of
these programs is optimal, in that it does no other computational work
other than that required to produce the output. As there is only one
way for a Start program to produce a specific output, the program that
produces this output and computes nothing is thus optimal, in that it
will run in a shorter amount of time than any other equivalent program
in this set.

This does not work if the program does not halt. However, for us to
apply a profile guided optimization, we are already making the
assumption that the program halts, and none of our profile guided
optimizations work if this assumption breaks.

In practice this optimization garners a massive speedup on any program
that isn't a series of print statements. This is still of limited use
in any useful setting, as most languages that exist in the wild have
the capacity for input to their programs.


\end{document}
